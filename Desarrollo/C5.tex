\section{Capítulo 5: Algoritmos Distribuidos}
\subsection{Conceptos Básicos}

\textbf{Definición}: Tipo de algoritmo que se ejecuta concurrentemente en una red de procesos, con información limitada sobre los demás procesos. Trabajando cooperativamente mediante paso de mensajes en lograr un objetivo común.

Ejemplos: Ordenamiento de eventos, mensajes u operaciones. Observación o monitoreo consistente. Elección de un líder o coordinador. Exclusión mutua distribuida.

\textbf{Caracterización de algoritmos distribuidos}: \begin{itemize}
    \item Tipo de topología: Anilo, árbol, grafo general (dirigido o no dirigido, total o parcialmente conectado)
    \item Modelos de computación: Estilo de interacción, modelo de sincronismo y restricciones temporales, modelo de fallas y fiabilidad
    \item Medidas de desempeño: Complejidad de mensajes (\# mensajes), complejidad de tiempo (\# rondas)
    \item Técnicas de diseño: Números de secuencia, relojes lógicos o vectoriales, marcas de tiempo para mensajes, contadores de mensajes, métodos de asentimiento (acknowledge), métodos de crédito, paso de testigo (token-passing)
\end{itemize}

\subsection{\textbf{Algoritmo de Eco-Chang}}

\textbf{Objetivo}: Visitar paralelamente todos los nodos de una red de procesos, representada como una topología de grafo general no dirigido, generando un spanning tree.

\textbf{Requisitos}: Topología de grafo general no dirigido y conexo. Se asume para el sistema o red de procesos un modelo asincrónico y sin fallas. Comunicación entre procesos es por paso de mensajes asincrónico y bidireccional. Cada nodo sólo conoce a los vecinos con quienes está directamente conectado. Existe un único nodo que comienza el algoritmo \textbf{(nodo iniciador)}

\textbf{\textcolor{blue}{Algoritmo de dos fases}}: \begin{itemize}
    \item \textbf{Exploración}: El nodo iniciador envía un mensaje a todos los nosdos de la red.
    \item \textbf{Eco}: Asentimiento (acknowledge) de los nodos que recibieron el mensaje, mandando un eco de vuelta al nodo iniciador.
\end{itemize}

\textbf{\textcolor{blue}{Comportamiento básico}}: \begin{itemize}
    \item Nodo \textbf{iniciador} envía el mensaje de \textbf{exploración} a sus vecinos.
    \item Si un nodo recibe el mensaje por primera vez, este \underline{memoriza el vecino de origen} y distribuye el mensaje a sus otros vecinos.
    \item Un nodo solo envía el mensaje de \textbf{eco} al vecino de origen cuando ha recibido mensajes de eco de todos sus otros vecinos.
    \item El algoritmo termina cuando el nodo iniciador recibe un mensaje de \textbf{eco} de todos sus vecinos.
\end{itemize}

\textbf{{\textcolor{purple}{Variables del algoritmo en cada proceso}}}: \begin{itemize}
    \item \textbf{comprometido}: Booleano que indica si está actibo el algoritmo en el proceso.
    \item \textbf{iniciador}: Booleano que indica si el proceso es iniciador.
    \item \textbf{n}: contador que indica número de mensajes recibidos por el proceso desde sus vecinos.
    \item \textbf{origen}: identifica al vecino desde el cual le llegó al proceso el \underline{primer} mensaje de exploración.
\end{itemize}

\textbf{Observación}: Inicialmente se asume que todo nodo no es iniciador, que no están comprometidos y que no han recibido mensajes.

\vspace{0.5em}

\textbf{Complejidad}: \textbf{Tiene complejidad 2e mensajes}, donde \textbf{e} es el número de aristas del grafo

Se usa paso de mensajes unidireccional y asincrónico. El algoritmo genera un spanning tree. Tiene aplicación para recolectar y difundir información ordenadamente en un grafo general. Se supone que el algoritmo se \underline{se inicia sólo en un nodo}.

\subsection{Algoritmos de elección}
\textbf{Objetivo}: Ponerse de acuerdo en una red de procesos, suponiendo que los procesos están únicamente identificados y es posible ordenarlos totalmente.

Ejemplos: Determinación de un único coordinador o líder para un grupo de procesos cooperativos. Elección de un coordinador para gestionar una transacción distribuida y comprometerla, etc.

\vspace{0.5em}

\textbf{Caracterización} \begin{itemize}
    \item \textbf{Topología}: Anillo, árbol, red totalmente conectada, grafo general (arbitrario).
    \item \textbf{Modelo del sistema}: Asincrónico vs Sincrónico, modelo de fallas
    \item \textbf{Criterios típicos de diseño}: Nodos o procesos con identificador único, usado como criterio de selección. Principio de extinción de mensajes para garantizar término del algoritmo en tiempo finito. Simetría: nodos iguales que ejecutan el mismo algoritmo y cualquiera es elegible. Nodos tienen conocimiento sólo de sus vecinos.
\end{itemize}